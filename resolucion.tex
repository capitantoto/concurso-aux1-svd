%% LyX 2.3.7 created this file.  For more info, see http://www.lyx.org/.
%% Do not edit unless you really know what you are doing.
\documentclass[english]{article}
\usepackage[T1]{fontenc}
\usepackage[latin9]{inputenc}
\usepackage{amsmath}
\usepackage{babel}
\begin{document}

\section*{Ejercicio 1}

Halle una descomposici�n en valores singulares de $A=\left[\begin{array}{cc}
2 & -1\\
2 & 2
\end{array}\right]$

\subsection*{Soluci�n}

Buscamos matrices ortogonales U, V y una matriz diagonal $\Sigma$
tales que $A=U\Sigma V^{T}$

Comenzamos por computar $B=A^{T}A$
\[
B=\left[\begin{array}{cc}
2 & 2\\
-1 & 2
\end{array}\right]\left[\begin{array}{cc}
2 & -1\\
2 & 2
\end{array}\right]=\left[\begin{array}{cc}
8 & 2\\
2 & 5
\end{array}\right]
\]

Luego, nos interesa conocer los autovalores $\lambda$ y autovectores
$v\neq0$ a derecha de B, tales que

\begin{align*}
Bv & =\sigma^{2}v\\
\left(B-\sigma^{2}I\right)v & =0\\
Cv & =0
\end{align*}

Como $v\ne0$, $C$ debe ser singular, con determinante $\left|\cdot\right|$
igual a cero. Construimos el \emph{polinomio caracter�stico }correspondiente:

\begin{align*}
p(x) & =\left|C\right|\\
 & =\vert B-xI\vert\\
 & =\left|\begin{array}{cc}
8-x & 2\\
2 & 5-x
\end{array}\right|\\
 & =\text{\ensuremath{\left(8-x\right)\left(5-x\right)}}-4\\
 & =40-8x-5x+x^{2}-4\\
 & =x^{2}-13x+36
\end{align*}

\begin{description}
\item [{Disgresi�n}] Para toda matriz cuadrada A de 2x2, su polinomio caracter�stivo
es $p\left(x\right)=\det A-\left(\text{Tr}A\right)x+x^{2}$. Para
toda matrix cuadrada de 3x3, su p.c. es $p\left(x\right)=\det A+\left(a_{ij}a_{ji}-a_{ii}a_{jj}\right)x+\left(\text{Tr}A\right)x^{2}-x^{3}$.
Pru�belo.
\end{description}
Las ra�ces de $p\left(x\right)$ resultan ser
\begin{align*}
\left\{ x_{0},x_{1}\right\}  & =\frac{13\pm\sqrt{13^{2}-4\times36}}{2}\\
 & =\frac{13\pm\sqrt{25}}{2}\\
 & =\left\{ 9,4\right\} 
\end{align*}

Y los valores singulares no-nulos de $\Sigma$ son las ra�ces de los
autovalores hallados, $\left\{ 3,2\right\} .$ Excelente! Como A es
de 2x2, U, $\Sigma$ y V tambi�n. En particular, ya podemos completar
$\Sigma=\left[\begin{array}{cc}
3 & 0\\
0 & 2
\end{array}\right]$, y no hace falta ``acolchonarla'' con columnas de ceros.

Sabemos adem�s que las columnas de $V$ son los autovectores (normalizados)
correspondientes a los autovalores de $B$. Computamos:
\begin{align*}
\left(B-9I\right)\left[\begin{array}{c}
x\\
y
\end{array}\right] & =\left[\begin{array}{c}
0\\
0
\end{array}\right]\\
\left[\begin{array}{cc}
-1 & 2\\
2 & -4
\end{array}\right]\left[\begin{array}{c}
x\\
y
\end{array}\right] & =\left[\begin{array}{c}
0\\
0
\end{array}\right]
\end{align*}

de donde se ve que un autovector cpte. a $\lambda_{1}=9$ es $v_{1}=\left[\begin{array}{c}
2\\
1
\end{array}\right]$. Procediendo an�logamente para $\left(B-4I\right)v=0$, encontramos
$v_{2}=\left[\begin{array}{c}
1\\
-2
\end{array}\right]$. Como la norma de ambos vectores es $\sqrt{5},$ podemos escribir

\[
V=V^{T}=\frac{1}{\sqrt{5}}\left[\begin{array}{cc}
2 & 1\\
1 & -2
\end{array}\right]
\]

Con un poco de manipulaci�n algebraica, podemos encontrar una expresi�n
sencilla para U:

\begin{align*}
A & =U\Sigma V^{T}\\
AV & =U\Sigma V^{T}V=U\Sigma\\
AV\Sigma^{-1} & =U\Sigma\Sigma^{-1}=U
\end{align*}

Es decir,

\begin{align*}
U & =AV\Sigma^{-1}\\
 & =\left[\begin{array}{cc}
2 & -1\\
2 & 2
\end{array}\right]\frac{1}{\sqrt{5}}\left[\begin{array}{cc}
2 & 1\\
1 & -2
\end{array}\right]\left[\begin{array}{cc}
1/3 & 0\\
0 & 1/2
\end{array}\right]\\
 & =\frac{1}{\sqrt{5}}\left[\begin{array}{cc}
3 & 4\\
6 & -2
\end{array}\right]\left[\begin{array}{cc}
1/3 & 0\\
0 & 1/2
\end{array}\right]\\
 & =\frac{1}{\sqrt{5}}\left[\begin{array}{cc}
1 & 2\\
2 & -1
\end{array}\right]
\end{align*}
Lo cual completa nuestra descomposici�n

\[
A=\frac{1}{\sqrt{5}}\left[\begin{array}{cc}
1 & 2\\
2 & -1
\end{array}\right]\left[\begin{array}{cc}
1/3 & 0\\
0 & 1/2
\end{array}\right]\frac{1}{\sqrt{5}}\left[\begin{array}{cc}
2 & 1\\
1 & -2
\end{array}\right]^{T}
\]


\section*{Ejercicio 2}
\begin{enumerate}
\item Escriba una funci�n en Python, $dvs\left(A\right)$, que devuelva
una terna U, S, VT con la descomposici�n en valores singulares de
A, donde U y V son las matrices $U,V^{T}$; y S es un vector tal que
$\Sigma=diag\left(S\right)$.
\item Compruebe que $dvs\left(\left[\begin{array}{cc}
2 & -1\\
2 & 2
\end{array}\right]\right)$ retorna una soluci�n equivalente a la que obtuvo en (1).
\item Genere 20 matrices $M_{1},\dots,M_{20}$ de 3x4 con entradas uniformemente
elegidas entre 0 y 1. Compute $dvs\left(A\right)$, y compruebe que
en todos los casos, el resultado sea equivalente a la implementaci�n
de referencia, $numpy.linalg.svd$
\end{enumerate}

\end{document}
